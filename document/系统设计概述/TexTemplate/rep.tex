\documentclass[a4paper,12pt]{article}


\usepackage{./SCUformat}
\usepackage{./cover}

\newcommand{\customtitle}{
    \begin{center}
        \Huge{hello \LaTeX}
        \vspace*{3cm}
    \end{center}
}


\begin{document}

\title{基于人工智能的AI}
\author{郑仕博}
% \authorEng{David Wang}
\adviser{杨波}
% \adviserEng{My Teacher}
\college{计算机学院}
% \collegeEng{Software College}
\major{计算机科学与技术}
% \majorEng{Software Engineering}
\date{\today}
\makecover

\tableofcontents
\clearpage

\section{引言}
\subsection{目的}
本软件设计文档描述了“基于生成式 AI 的个性化文创图像作品设计系统”的
架构与系统设计。面向开发、测试、维护本项目的工程人员及项目管理者,作为
技术实现和系统集成的参考依据。
\subsection{项目范围}
该软件旨在利用生成式 AI 技术解决个性化文创产品供给不足的问题,核心
功能包括:根据用户输入的文本和指定的位置生成创意图像,或编辑现有图像中
的文本。重点目标是实现中文字符的高精度渲染,便于游客与文创从业者快速创
作独特图像作品,助力文旅融合与传播。
\subsection{文档概览}
第 1 章介绍目的、范围、参考资料和术语;第 2 章提供系统概览;第 3 章
详细阐述系统架构;第 4 章描述数据设计;第 5 章介绍各组件设计;第 6 章讲
解人机界面设计;第 7 章为需求矩阵;第 8 章为附录。
\subsection{参考资料}
信息来源于网页\url{https://www.sohu.com/a/823541100_234564}。技术细节参考了
AnyText、TextDiffuser、DDPM 等文献。文档结构参考与\url{https://github.com/SPM-PSP/SPM-PSP-Course-github/blob/main/SDD_Template.pdf}。
\subsection{术语与缩略语}
AI(人工智能)、SDD(软件设计文档)、VAE(变分自编码器)、UNet
(网络结构)、Stable Diffusion(SD,扩散模型)、AnyText(生成式模型)、
Text-control Diffusion Pipeline 、 Auxiliary Latent Module 、 Text Embedding
Module、Gradio(UI 库)、Prompt(文本提示)、OCR(光
学字符识别)、FID(图像质量指标)、CFG-Scale(无
分类引导因子)、eta(扩散采样参数)等术语在文中根据需要进一步解释。

\section{系统概览}
本系统是一个利用生成式 AI 的图像创作工具,支持文本生成图像和图像内
文字编辑,专注于中文字符的精准渲染。系统基于 AnyText 并通过Google提出的Dreambooth
方法微调 Stable
Diffusion 模型,通过 Web 界面(Gradio)与用户交互,后端使用 Python 与深
度学习框架实现,支持 Docker 部署。系统的目的是解决文创产品同质化问题,赋
能个体创作。

\section{系统架构}
\subsection{架构设计}
系统分为三层:用户界面层(Gradio 实现)、应用逻辑层。

用户界面层:负责输入(文本、图像、参数、坐标绘制)与结果展示。

应用逻辑层:解析输入、格式化参数、调用模型、处理输出与数据管理。以 AnyText 为核心,包括三大子模块:

\begin{enumerate}
    \item 文本嵌入模块(Text Embedding Module)
    \item 辅助潜变量模块(Auxiliary Latent Module)
    \item 文本控制扩散管道(Text-control Diffusion Pipeline)
\end{enumerate}
\subsection{分解描述}
Text Embedding Module:接收用户提示词和需渲染文本,对需要生成的文字
用占位符占位,生成对应字形图,用OCR提取特征后替换占位符,然后传入到Clip编码器。

Auxiliary Latent Module:掩码、字形图,通过卷积处理
生成与扩散模型匹配的空间向量。

Diffusion Pipeline:以初始噪声为起点,联合文本嵌入与空间特征逐
步去噪生成图像潜变量,最后 VAE 解码。

Gradio UI:提供文本输入、图像上传、画布交互、参数调节、结果展示等供用户操作的界面
展示等功能。
\subsection{设计原理}
采用 AnyText + Stable Diffusion v1.5 架构,针对中文文本渲染难题,结合字形
信息与位置控制,以Realistic\_Vision\_V4.0 作为底模保证图像质量和真实性,通过对AnyText框架进行中文语料微调
提升中文文字的准确率,用Dreambooth 对扩散模型进行微调,采取Gradio 快速构建 用户界面。

\section{数据设计}
\subsection{数据说明}
输入数据包括提示词(文本)、需渲染文本、位置坐标、参考图像(可选)、
控制参数;训练后的权重以ckpt文件存储,约5.73GB,训练数据包含两类:
\begin{enumerate}
    \item AnyWord-3M 标注数据(JSON 格式),筛选后约400k张,用于文字渲染的训练;

    \item 文创图像+文本描述(TXT 格式),约1k张,用于风格微调和物品的学习。
\end{enumerate}

输出图像保存在服务器并将图像、debug信息(可选)返回给用户。
\subsection{数据字典}
user\_prompt:字符串

text\_to\_render:字符串列表

position\_data:坐标列表

edit\_mask:掩码图像/张量

reference\_image:上传图像

control\_params:参数字典,如\{'cfg\_scale': 7.5\}

generated\_image:最终生成图像

training\_data\_1:AnyWord-3M JSON 结构

training\_data\_2:TXT 列表与对应图像

model\_weights:模型权重文件

glyph\_image、text\_embedding、auxiliary\_latent、image\_latent:中间张
量

hehe98/wenchuang: 项目镜像,详情见dockerhub

wenchuang.ckpt: 模型权重文件

strength: 文字渲染控制强度,可以为 0 即不使用文字渲染

CFG-Scale: 文字控制强度,低的话会导致生成图像与描述不符合,高的
话图像会不自然

eta:风格多样性,1 表示启用(更具变化),0 不启用(更保守)

\section{组件设计}
主要功能以组件化方式组织,核心函数如下:

generate\_image:解析提示词,生成字形图和文本嵌入,调用辅助模块生成
空间信息,联合生成潜变量图像,再解码输出;

edit\_image:编码参考图像,加入掩码噪声生成初始状态,调用编辑流程生
成新图像并融合原图。

每个过程中的局部变量包括潜变量,预测噪声,注意力图等张量,模块间依
次传递处理。

\section{人机界面设计}
\subsection{界面概览}
提供 Web 端界面,两种主要操作模式:
\begin{enumerate}
    \item 文到图像的生成:输入提示词,并将需要渲染的文本用“”标注,
    可以通过画布绘制文本位置、拖框选择文本位置或随机选择文本位
    置;
    \item 图片文字编辑,手动掩盖需要修改区域,输入文本并进行编辑。

\end{enumerate}

界面上有说明、参数设置、文本输入框、模式选择、文字位置标注、样例(
点击即可)、运行按钮、图片结果展示和加强训练的物品,用户可调整 CFG-Scale、Steps 等参数,查看结果并保存。
\subsection{界面截图}
详情见图1,图2.
\begin{figure}[htbp] % 控制图片浮动位置的选项 (h: here, t: top, b: bottom, p: page)
    \centering % 让图片在页面上居中显示
    \includegraphics[width=1\textwidth]{Image/UI_1.png} % 插入图片并设置宽度
    \caption{这是图片的标题} % 添加图片标题
    \label{fig:logo} % 添加标签,方便在文中引用,例如 \ref{fig:logo}
\end{figure}
\begin{figure}[htbp] % 控制图片浮动位置的选项 (h: here, t: top, b: bottom, p: page)
    \centering % 让图片在页面上居中显示
    \includegraphics[width=1\textwidth]{Image/UI_2.png} % 插入图片并设置宽度
    \caption{这是图片的标题} % 添加图片标题
    \label{fig:logo} % 添加标签,方便在文中引用,例如 \ref{fig:logo}
\end{figure}

\subsection{界面控件与操作}
\noindent 包括:

说明文本框

文本输入框(Prompt)

位置选择方式(单选按钮)

绘制画布(支持自由绘制、矩形、掩码)

参数调节控件(滑动条/输入框)

“运行”按钮

图像展示区域

图片上传控件

示例加载按钮

参考生成物品展区

加强训练物品展区

\noindent 操作:

用户可点击说明查看使用须知,点击参数调整控件调整参数,在文本输入框输入文字
进行提示词输入,点击运行进行生成,点击样例进行生成,在模式选择框选择模式。
\section{需求矩阵}
详情见表1。
\begin{table}[htbp]
    \centering
    \begin{tabular}{|p{4cm}|p{10cm}|}
      \hline
      \textbf{需求} & \textbf{组件} \\
      \hline
      文本输入(提示词)
        & 文本输入框 \\
      \hline
      图像上传
        &  图片上传控件(或者绘制画布)\\
      \hline
      指定文字位置
        & 绘制画布 \\
      \hline
      参数调节
        & 参数调节控件 \\
      \hline
      结果预览
        & 图像展示区域 \\
      \hline
      保存分享
        & 图像展示区域 \\
      \hline
      Debug
        & 图像展示区域和参数调节控件 \\
      \hline
      模式选择
        & 图片上传控件 \\
      \hline
      示例与指导
        & 说明文本框 \\
      \hline
    \end{tabular}
    \caption{功能与需求表}
\end{table}
  

\section{APPENDICES}
详见材料中的“项目注意事项”文档。


\end{document}
